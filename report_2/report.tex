\documentclass[11pt]{article}
\usepackage[utf8]{inputenc}
\usepackage[T1]{fontenc}
\usepackage{minted}
\usepackage{graphicx}
\usepackage{hyperref}

\author{Student: Sean Wang, szw87 \\ Professor: Mohit Tiwari, Antonio Espinoza \\ Department of Electrical \& Computer Engineering \\ The University of Texas at Austin}
\date{\today}
\title{EE379K Enterprise Network Security Lab 2 Report}
\hypersetup{
 pdfauthor={Student: Sean Wang, szw87 \\ Professor: Mohit Tiwari, Antonio Espinoza \\ Department of Electrical \& Computer Engineering \\ The University of Texas at Austin},
 pdftitle={EE379K Enterprise Network Security Lab 2 Report},
 pdfkeywords={},
 pdfsubject={},
 pdfcreator={},
 pdflang={English}}

\begin{document}

\maketitle
\section*{Part 3 - Orchestration}
\subsection*{3a - Orchestration with Kubernetes}
\subsubsection*{Docker applications}
Running the simple PHP and MySQL server example with
\begin{minted}{bash}
  $ docker-compose up
\end{minted}
sets up both the web-service and SQL DB are on localhost. The web-service can be accessed through \verb|localhost:8000|,
which maps internally to port 80 inside the container. Additionally, the web-service uses port 3306 to access the SQL DB,
while port 8082 is exposed to the host. The contents seen on the homepage, \verb|http://localhost:8000|, is due to
\verb|src/index.php|. \\
By going into \verb|docker-compose.yml| and changing the port mapping from \verb|8000:80| to \verb|9000:80|, as shown in
Figure~\ref{fig:port}, the web-server can now be accessed at \verb|http://localhost:9000|, since this changed what port is
exposed to the host machine and maps it to the internal port.
\clearpage
\begin{figure}[h!]
  \centering
  \includegraphics[width=1\linewidth]{./docker_port.png}
  \caption{\label{fig:port}
  Changing port mapping inside docker-compose.yml}
\end{figure}
\subsubsection*{Kubernetes}
After tagging and pushing the web-service image to the microk8s registry, then the following commands are used to run the
web-application in kubernetes:
\begin{minted}{bash}
  $ microk8s.kubectl apply -f webserver.yaml
  $ microk8s.kubectl apply -f webserver-svc.yaml
  $ microk8s.kubectl apply -f mysql.yaml
  $ microk8s.kubectl apply -f mysql-svc.yaml
\end{minted}
Then, the different namespaces, shown in Figure~\ref{fig:pods} and Figure~\ref{fig:svcs}, are seen under the
\verb|NAMESPACE| column in each of the outputs of the following commands:
\begin{minted}{bash}
  $ microk8s.kubectl get pods --all-namespaces
  $ microk8s.kubectl get services --all-namespaces
\end{minted}
\begin{figure}[htbp]
  \centering
  \includegraphics[width=1\linewidth]{./get_pods_1.png}
  \caption{\label{fig:pods}
  Output of microk8s.kubectl get pods --all-namespaces}
\end{figure}
\begin{figure}[htbp]
  \centering
  \includegraphics[width=1\linewidth]{./get_services_1.png}
  \caption{\label{fig:svcs}
  Output of microk8s.kubectl get services --all-namespaces}
\end{figure}
For example, the \verb|default| namespace refers to the default namespace for objects without any specified
namespace. Additionally, Kubernetes creates the \verb|kube-system| namespace, and it includes pods and services
like the dashboard.~\cite{namespace}\\
In the \verb|webserver.yaml| file, there are specifications on how many instances of each application to deploy
under \verb|spec/replicas|:
\begin{minted}{yaml}
  apiVersion: apps/v1
  kind: Deployment
  ...
  spec:
    replicas: 3
  ...
\end{minted}
This value can be changed to change the number of instances of web-servers. For example, if it was changed to 2,
then the output of the \verb|microk8s.kubectl get| commands would be the following:
\begin{figure}[htbp]
  \centering
  \includegraphics[width=1\linewidth]{./get_pods_2.png}
  \caption{\label{fig:pods2}
  New output of microk8s.kubectl get pods --all-namespaces}
\end{figure}
\begin{figure}[htbp]
  \centering
  \includegraphics[width=1\linewidth]{./get_services_2.png}
  \caption{\label{fig:svcs2}
  New output of microk8s.kubectl get services --all-namespaces}
\end{figure}
\subsubsection*{RBAC}
For Role Based Access Control, first a service account and role need to be created and then
bound together. Then, the following command can be used to set up and run the Kubernetes
Dashboard:
\begin{minted}{bash}
  $ microk8s.kubectl -n kube-system
      edit service kubernetes-dashboard
\end{minted}
The type must be changed to \verb|NodePort| and the exposed port is given under \verb|ports/nodePort|:
\begin{minted}{yaml}
  ...
  spec:
    clusterIP: 10.152.183.68
    ports:
    - nodePort: 32388 # port num
      port: 443
      protocol: TCP
      targetPort: 8443
    selector:
      k8s-app: kubernetes-dashboard
    sessionAffinity: None
    type: NodePort # change from ClusterIP to NodePort
  ...
\end{minted}
Then, once a secret token is obtained, the dashboard can be opened and a list of all the pods in the
default namespace can be seen, like in Figure~\ref{fig:def_dash}. Only these pods are shown because
the namespace of the user-sa account is set to default, which is specified in the
\verb|sa-role-bind.yaml| file:
\begin{minted}{yaml}
  ...
  subjects:
  - kind: ServiceAccount
    name: user-sa
    namespace: default
  ...
\end{minted}
In order to create another service account that can access just the kube-system namespace, a new
service account must be initialized. This can be done by first creating a service account and then
making slight modifications to the \verb|user-role.yaml| and \verb|sa-role-bind.yaml| files, as can
be seen in \verb|part3/kube-role.yaml| and \verb|part3/kube-sa-role-bind.yaml|, respectively. After
creating the new service account, login to the dashboard with the token for the new account and now
nothing can be seen in the \verb|default| namespace, but the pods in the \verb|kube-system| namespace
are now visible on the dashboard, as shown in Figure~\ref{fig:kube_dash}. The sequence of commands
to set this up are as follows:
\begin{minted}{bash}
  $ microk8s.kubectl create serviceaccount kube-sa --namespace kube-system
  $ microk8s.kubectl apply -f kube-role.yaml
  $ microk8s.kubectl apply -f kube-sa-role-bind.yaml
\end{minted}
\begin{figure}[htbp]
  \centering
  \includegraphics[width=1\linewidth]{./def_dash.png}
  \caption{\label{fig:def_dash}
  Dashboard view of default namespace visible to user-sa}
\end{figure}
\begin{figure}[htbp]
  \centering
  \includegraphics[width=1\linewidth]{./kube_dash.png}
  \caption{\label{fig:kube_dash}
  Dashboard view of kube-system namespace visible to kube-sa}
\end{figure}


\bibliography{bibliography}
\bibliographystyle{ieeetr}
\end{document}
